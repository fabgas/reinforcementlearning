% !TEX TS-program = pdflatex
% !TEX encoding = UTF-8 Unicode

% This is a simple template for a LaTeX document using the "article" class.
% See "book", "report", "letter" for other types of document.

\documentclass[11pt]{article} % use larger type; default would be 10pt

\usepackage[utf8]{inputenc} % set input encoding (not needed with XeLaTeX)

%%% Examples of Article customizations
% These packages are optional, depending whether you want the features they provide.
% See the LaTeX Companion or other references for full information.

%%% PAGE DIMENSIONS
\usepackage{geometry} % to change the page dimensions
\geometry{a4paper} % or letterpaper (US) or a5paper or....
% \geometry{margin=2in} % for example, change the margins to 2 inches all round
% \geometry{landscape} % set up the page for landscape
%   read geometry.pdf for detailed page layout information

\usepackage{graphicx} % support the \includegraphics command and options

% \usepackage[parfill]{parskip} % Activate to begin paragraphs with an empty line rather than an indent

%%% PACKAGES
\usepackage{booktabs} % for much better looking tables
\usepackage{array} % for better arrays (eg matrices) in maths
\usepackage{paralist} % very flexible & customisable lists (eg. enumerate/itemize, etc.)
\usepackage{verbatim} % adds environment for commenting out blocks of text & for better verbatim
\usepackage{subfig} % make it possible to include more than one captioned figure/table in a single float
% These packages are all incorporated in the memoir class to one degree or another...

%%% HEADERS & FOOTERS
\usepackage{fancyhdr} % This should be set AFTER setting up the page geometry
\pagestyle{fancy} % options: empty , plain , fancy
\renewcommand{\headrulewidth}{0pt} % customise the layout...
\lhead{}\chead{}\rhead{}
\lfoot{}\cfoot{\thepage}\rfoot{}

%%% SECTION TITLE APPEARANCE
\usepackage{sectsty}
\allsectionsfont{\sffamily\mdseries\upshape} % (See the fntguide.pdf for font help)
% (This matches ConTeXt defaults)

%%% ToC (table of contents) APPEARANCE
\usepackage[nottoc,notlof,notlot]{tocbibind} % Put the bibliography in the ToC
\usepackage[titles,subfigure]{tocloft} % Alter the style of the Table of Contents
\renewcommand{\cftsecfont}{\rmfamily\mdseries\upshape}
\renewcommand{\cftsecpagefont}{\rmfamily\mdseries\upshape} % No bold!

%%% END Article customizations
\usepackage{amsmath} % package mathématique
%%% The "real" document content comes below...

\title{Court terme vs long terme}
%\author{The Author}
\date{} % Activate to display a given date or no date (if empty),
         % otherwise the current date is printed 

\begin{document}
\maketitle

\section{Définitions}
\subsection{Episodes}
\subsection{Récompenses}
Le choix des actions par l'agent doit être guidée par la récompense espérée, mais pas uniquement la récompense à court terme qui n'est pas suffisante.
L'apprentissage d'un agent doit tenir compte de l'écart qu'il peut y avoir entre une récompense à court terme et une récompense à long terme.
Rechercher la récompense à court terme peut nous éloigner d'une récompense à long terme. Ainsi, lorsque face à une montagne on choisit le contournement de celle -ci, le chemin peut s'avérer plus long. 
Pour effectuer la balance entre ces deux horizons il faut définir une mesure (car seuls les chiffres sont pertinents).
La définition courrament admise est la suivante :
\begin{equation}
G_t = R_{t+1} + R_{t+2} + ... R_{\infty}	
\end{equation}
Elle établit un lien entre  la récompense espérée à l'instant t comme la somme des récompenses futures.

Cependant, plusieurs problèmes apparaîssent immédiatement : 
\begin{itemize}
	\item la récompense peut être infini si l'épisode n'a pas de fin.
	\item une récompense très lointaine à le même poids qu'une récompense à court terme. Or on souhaite que l'agent aille le plus vite possible vers les meilleurs récompenses 
\end{itemize}

Pour corriger ces points, il est nécessaire d'ajouter un facteur de "discount" $\gamma$ qui va atténuer l'effet des récompenses à long terme vis à vis des récompenses à court terme. 
\begin{equation}
G_{t} = \sum_{n=0}^{\infty} \gamma^n R_{t+n+1} 
\end{equation}
où $\gamma \in [0,1]$. Avec ce facteur, les récompenses à long terme participent de moins en moins à la récompenses espérée. 

Reste à construire des algorithmes autour de cette définition pour correctement paramétré l'agent.

\subsection{La fonction de valeur}

L'agent suite à une action va atteindre un certain état d l'environnement. Mais comment choisir sa destination et donc l'action ? Il faut pour cela déterminer à quel point il peut être intéressant d'être dans un certain état. 
L'agent choisira une action parce que elle l'emmènera vers un état qui lui sera favorable en terme de récompense.
Reste à quantifier cette idée d'état favorable. Il s'agit de la récompense espérée en étant dans l'état $s$ en suivant une politique $\pi$
\begin{align}
v_\pi (S) &= E_\pi \{G_t |s\} \\
		&= \sum_{a} \pi(s,a) \sum_{s'}p(s,r|s',a)[r+ \gamma v(s')]
\end{align}
C'est une équation récursive ($\propto v_\pi$) , elle décrit donc la dynamique d'un épisode complet en partant de l'état $s$. Elle inclue la référence à la récompense à long terme. $s'$ est l'état précédent l'état $s$ et la transition entre $s'$ et $s$ se fait par l'action $a$.
$p(s,r|s',a)$ décrit la possibilité de passer de l'état $s'$ à l'état $s$ à l'aide de l'action $a$ en obtenant une récompense $r$. La probabilité pouvant bien sur être nulle. La politique $\pi(s,a)$ fournie la liste des actions disponible en étant dans l'état $s$.  
\section{références}
https://medium.com/@m.alzantot/deep-reinforcement-learning-demysitifed-episode-2-policy-iteration-value-iteration-and-q-978f9e89ddaa
\end{document}
