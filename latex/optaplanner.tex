% !TEX TS-program = pdflatex
% !TEX encoding = UTF-8 Unicode

% This is a simple template for a LaTeX document using the "article" class.
% See "book", "report", "letter" for other types of document.

\documentclass[11pt]{article} % use larger type; default would be 10pt

\usepackage[utf8]{inputenc} % set input encoding (not needed with XeLaTeX)

%%% Examples of Article customizations
% These packages are optional, depending whether you want the features they provide.
% See the LaTeX Companion or other references for full information.

%%% PAGE DIMENSIONS
\usepackage{geometry} % to change the page dimensions
\geometry{a4paper} % or letterpaper (US) or a5paper or....
% \geometry{margin=2in} % for example, change the margins to 2 inches all round
% \geometry{landscape} % set up the page for landscape
%   read geometry.pdf for detailed page layout information

\usepackage{graphicx} % support the \includegraphics command and options

% \usepackage[parfill]{parskip} % Activate to begin paragraphs with an empty line rather than an indent

%%% PACKAGES
\usepackage{booktabs} % for much better looking tables
\usepackage{array} % for better arrays (eg matrices) in maths
\usepackage{paralist} % very flexible & customisable lists (eg. enumerate/itemize, etc.)
\usepackage{verbatim} % adds environment for commenting out blocks of text & for better verbatim
\usepackage{subfig} % make it possible to include more than one captioned figure/table in a single float
% These packages are all incorporated in the memoir class to one degree or another...

%%% HEADERS & FOOTERS
\usepackage{fancyhdr} % This should be set AFTER setting up the page geometry
\pagestyle{fancy} % options: empty , plain , fancy
\renewcommand{\headrulewidth}{0pt} % customise the layout...
\lhead{}\chead{}\rhead{}
\lfoot{}\cfoot{\thepage}\rfoot{}

%%% SECTION TITLE APPEARANCE
\usepackage{sectsty}
\allsectionsfont{\sffamily\mdseries\upshape} % (See the fntguide.pdf for font help)
% (This matches ConTeXt defaults)

%%% ToC (table of contents) APPEARANCE
\usepackage[nottoc,notlof,notlot]{tocbibind} % Put the bibliography in the ToC
\usepackage[titles,subfigure]{tocloft} % Alter the style of the Table of Contents
\renewcommand{\cftsecfont}{\rmfamily\mdseries\upshape}
\renewcommand{\cftsecpagefont}{\rmfamily\mdseries\upshape} % No bold!

%%% END Article customizations

%%% The "real" document content comes below...

\title{Integration d'Optaplanner dans ABValue}
\author{The Author}
%\date{} % Activate to display a given date or no date (if empty),
         % otherwise the current date is printed 

\begin{document}
\maketitle

\section{Ijntroduction}

Your text goes here.

\subsection{Objectif}
Le but de ce travail sera de vérifiier la possibilité d'intégrer la librairie OptaPlanner dans la stack technique ABValue, à minima de les faire interagir.
Pour le démontrer , un cas d'usage fonctionel devra être implémenter.
Ce cas d'usage sera suffisament simple pour ne pas être consommateur de temps mais d'une complexité suffisante pour initier un apport pour le produit ABValue.

Pour cela je propose d'appliquer Optaplanner à la planification d'une journée. Cette planification devra se faire au processus - shift en tentant de traiter une charge intiale fournie par shift.
Elle devra s'appuyer sur les données initiales suivantes :
\begin{itemize}
 \item les opérateurs ont été planifié sur un shift (peu importe le processus)
 \item les opérateurs ont une compétence pour chacun des processus (avec une valeur par default)
  \item les processus ont un volume par shift 
 \item les processus ont une productivité managériale.
\end{itemize}

Quelques règles :
\begin{itemize}
 \item la planification se fera sur des créneaux d'une heure 
 \item l'opérateur pourra être planifié sur plusieurs processus dans une journée
 \item l'opérateur ne pourra être affecté qu'a un processus à la fois dans un créneau (règles  dure).
 \item le but est de traiter tous les objets d'une journée (regles soft)
\end{itemize}
\end{document}
